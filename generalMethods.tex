
 
% ========== Chapter on General Methods
 
\chapter {General methods and background used throughout the dissertation} \label{chap:generalMethods}

For all subdural grid electrode work, human subjects are implanted at Harborview Medical Center (Seattle, WA), with electrocorticographic (ECoG) grids (2.3 mm exposed diameter, 10 mm spacing, Ad-tech Medical, Racine, WI, USA) for acute clinical monitoring of intractable epilepsy prior to surgical resection (Figure \ref{fig:gmBrainSchematic}). ECoG grid placement are determined solely based on clinical needs without consideration of research benefits. We conduct all DCS studies after subjects were back on their anti-epileptic medications, after approximately one week of clinical monitoring. All patients give informed consent under a protocol approved by the University of Washington Institutional Review Board.

For work involving deep brain stimulator patients, human subjects are chronically implanted at the University of Washington Medical Center (Seattle, WA), with Medtronic 4 contact DBS leads (Medtronic, Minneapolis, MN, USA) for treatment of Parkinson’s and Essential tremor disease. During the course of our intraoperative work, they are implanted with 8 contact subdural strip electrodes (2.3 mm exposed diameter, 10 mm spacing). A Cadwell clinical system (Cadwell, Kennewick, WA, USA) is used for somatosensory evoked potential (SSEP) and motor evoked potential (MEP), as the TDT system is not capable of generating the required output current to consistently elicit responses. 

The DBS electrodes that subjects are implanted with are frequently the Medtronic 3387 lead (Medtronic, Minneapolis, MN), which have 4 cylindrical contacts with 1.5 mm spacing between contacts, 1.5 mm contact height, and 1.26 mm contact diameter. 


\section{Cortical reconstructions}

We perform cortical reconstructions (Figure \ref{fig:brainSchematic}) using previously described techniques \cite{Blakely2009,Hermes2010a,Wander2016}. In brief, we co-register the electrode locations as localized by a post-operative CT scan with a preoperative MRI of the brain to build patient specific models. 

\section{Data acquisition and stimulation}

Neural data are acquired at 12 kHz using a Tucker Davis Technologies (TDT) System 3 with the RZ5D and PZ5 Neurodigitizer (Tucker Davis Technologies, Alachua, Florida, USA). We deliver stimulation through the TDT IZ2H-16 stimulator and LZ48-400 battery pack (Tucker Davis Technologies) with biphasic, bipolar, constant current stimulation trains. 

\section{Determination of sensory cortex in epilepsy patients}

In order to localize sensory cortical regions in epilepsy patients, we first consult the clinical notes of the epilepsy monitoring team at the hospital. Based off of any prior somatosensory evoked potential (SSEP) mapping, we decide on a subset of the cortical electrodes to test for sensory percepts. 

\section{Determination of sensory and motor cortex in DBS patients  }

IIn order to localize sensory cortical regions in DBS patients, we perform somatosensory evoked potential (SSEP) screening using a Cadwell clinical system.  Once the electrode strip is placed, we use the phase reversal technique during median nerve stimulation (SSEPs) to locate the central sulcus \cite{Cedzich1996a}, from which we have knowledge of the different cortical regions covered by particular strip electrodes. 

\section{Haptic stimulation}

We apply haptic feedback with digital touch probes (Karolinska Institute, and our own custom probes) that time stamped the deflection, and touched the cutaneous region where subjects localized the DCS percepts. Characterization of the digital touch probes is discussed in Chapter \ref{chap:responseTiming}. An audio signal presented to the researcher via headphones but which is inaudible to the subject, cues the experimenter to apply the haptic feedback. During experiments where we conduct both DCS and haptic stimuli, the subjects are blindfolded to avoid any confounders from simultaneous visual processing. 


\begin{figure}[ht]
	\centering
	\includegraphics[width=1\textwidth]{figures/schematic/grid_overview_pictures_3ada8b_v2}
	\caption[Subject implanation, cortical reconstruction, and electrode registration]{a) Craniotomy window prior to electrode implantation. The dura has visibly been cut, and the cortical surface is covered by the arachnoid and pia mater. b) 64 contact electrode grid placed subdurally. c) Dura sewn d) Cortical regions, as well as the approximate location of the central sulcus indicated. e) CT image illustrating the electrode grid after implantation. f) 3D Reconstructed cortical surface with electrode locations, using preoperative MRI scans and the postoperative CT images.}
	\label{fig:gmBrainSchematic}
\end{figure}