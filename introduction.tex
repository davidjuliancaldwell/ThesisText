
% ========== Introduction Chapter
 
\chapter {Introduction}

\section {Thesis}

Use of direct electrical stimulation (DES) within a neural engineering context to restore sensation and enhance connectivity can be accomplished through in-dwelling electrocorticography (ECoG) electrodes.  Successful translation of human DES will require substantial gains in knowledge of its spread through cortex, how it activates populations of neurons and affects neuronal plasticity, and interpretation of the underlying cortical dynamics in response to stimulation. These elements are essential to properly engineer closed-loop functional neuroprosthetics for sensory restoration and plasticity enhancement.  

\section{Introduction}

Damage to the nervous system due to stroke, spinal cord injury, and limb loss accounts for significant sensory and motor deficits. Despite the diversity of reasons behind cortical injury, targeted cortical stimulation may be able to help restore both motor and sensory function.  

My dissertation will shed light on different aspects of engineered cortical stimulation, from understanding how it spreads through cortex, how to interpret neural signals during ongoing stimulation, the behavioral and neural effects of stimulation, and how different stimulation protocols work in humans. These studies may have future application in informing clinical trials and neuroprosthetic device design. 

\section{Clinical Need}

There are millions of individuals who are disabled due to stroke. It is estimated in the US alone for the year 2016, that healthcare and economic costs related to stroke disability totaled \$34 billion, and that stroke is a leading cause of serious long-term disability \cite{CDC2015}. 50-70\% of stroke survivors reach functional independence, but 15-30\% of survivors are permanently disabled \cite{Lloyd-Jones2010}. Therapies using targeted activity neuromodulation may be able to help restore motor recovery \cite{HarveyRL.WinsteinCJ.EverestTrial2009}, but the biological effects of cortical stimulation is not well understood, and the parameters for potentially effective stimulation protocols need further development.  

The motor output deficits associated with stroke are discussed more frequently than sensory deficits, but sensory feedback is critical for the planning and execution of voluntary movements \cite{Bolognini2016}. Individuals suffering from somatosensory and visual deficits following stroke have worse motor recovery, highlighting the complex interplay between sensory and motor cortices. This motivates the exploration of techniques applicable to both regions. Somatosensory cortex is important for integrating sensory and motor signals and motor learning as well \cite{Borich2015}. This integration of sensory and motor information, along with all of the connections between sensory and motor cortical regions \cite{Ackerley2015}, illustrates that engineering solutions should consider the complexities of the two highly interconnected regions. 

Additionally, it is estimated that 5.4 million Americans are living with paralysis, with an estimated 41.8\% of people living with paralysis unable to work \cite{Foundation2013}. The restoration of sensation is a priority for prosthetics users \cite{Biddiss2007a} as well as potential BCI end users such as individuals with paralysis \cite{Collinger2013,Anderson2004}. Sensory feedback to cortex would enhance the efficacy of a prosthetic arm to aid with independent tasks, or help an individual better interpret data from body mounted sensors. The lack of sensory feedback in many existing brain computer interfaces (BCIs) may limit performance \cite{Delhaye2016,Bensmaia2014a}. Indeed, integration of somatosensory feedback into BCIs has been demonstrated to improve task performance with BCIs\cite{Pistohl2015,Dadarlat2015,Klaes2014,Suminski2010,Schiefer2016a}.This points to the importance of understanding how to appropriately engineer stimulation to restore sensation in BCI devices. 
 
One aspect of engineering stimulation for both connection enhancement and sensation restoration is understanding the source of signals which may be used to trigger efficacious stimulation, as well be influenced by stimulation. 

\section{Electrocorticography and DBS Background}

Electrocorticography (ECoG) is used clinically as a recording modality for diagnosing specific spatial regions of focal epilepsy onset in individuals suffering from medically intractable epilepsy. By using invasive monitoring, the origins of seizures can be identified, and subsequent surgical removal of the seizure foci can reduce the frequency of or eliminate seizures. After surgical resection, approximately 50\% or greater of patients experience significantly improved seizure control following surgical treatment \cite{Englot2014}. For monitoring, patients are routinely implanted for 1 to 2 weeks with electrodes either directly on top of the dura (epidural), beneath the dura (subdural), or implanted in cortex (depth electrodes, or stereo electroencephalography (sEEG)). The term intracranial EEG, or iEEG, is often used to describe all implanted electrodes. We will use the term ECoG electrodes in this thesis to encompass surface as well as penetrating depth electrodes. Following electrode implantation, patients remain in the hospital under clinical monitoring by a team of neurologists and epilepsy technicians, until the clinical team has collected enough data to precisely localize the focal seizure zones for surgical resection. 

To complement the passive recording of epileptic events, direct electrical stimulation (DES) \cite{Vincent2016} (or when applied particularly to cortex, known as direct cortical stimulation (DCS) \cite{Giussani2010}, or direct electrical cortical stimulation (DECS) through ECoG electrodes is commonly performed for clinical mapping purposes, both intraoperatively and during the patients’ clinical observation. Here, the clinical team electrically stimulates different brain regions to delineate regions of cortex important for language, motor, and sensory function. By stimulating particular brain areas and observing the effects by querying the patient, the clinical team can avoid resecting areas important for cognitive function and preserve these functions in an individual after surgical resection. The combination of recording and mapping through stimulation enables the clinical team to be best informed when making clinical decisions regarding maximizing the chances of reducing or eliminating seizures, while maintaining cortical function. Clinical teams perform stimulation of both cortical and subcortical structures and pathways, and we consider DES here to refer to general electrical stimulation of any brain region through implanted electrodes, while we consider DCS a subcategory specifically describing stimulation of surface gray matter. 

DES for clinical uses goes beyond delineating cortical regions of activity, as deep brain stimulation (DBS) is a therapy currently being used for therapeutic treatment of movement disorders. Electrodes similar to those used for sEEG are implanted into deep brain structures, and stimulation helps ameliorate clinical symptoms. The space of DBS research is vast, and this thesis does not exhaustively address DBS. I discuss stimulation through DBS as a demonstration of clinical stimulation through implanted electrodes, as well as draw from current research in this field for future directions of stimulation through all types implanted electrodes. 


\section{The Electrical-Neural Interface}

\subsection{Electrodes}

Current clinically used ECoG electrodes are often made of platinum or stainless steel, and can range from circular 1.5 mm diameter contacts with 4 mm spacing for “micro”-ECoG electrodes, to 2.3-3 mm diameter contacts with 10 mm spacing for “macro”-ECoG electrodes \cite{Chang2015a}. They are often embedded in a silicone sheet. Depth electrodes are frequently comprised of platinum, with cylindrical contacts, and can be inserted with or without stereotactic guidance. These are commonly used to localize seizures coming from deep brain structures, such as the hippocampus. DBS electrodes are similar to depth electrodes as they are linear probes with cylindrical contacts, although they can be of smaller diameter, with tighter electrode spacing and fewer contacts. The principles of stimulation recording discussed here refer to macroscale implanted electrodes in general. 

\subsection{Stimulation}

Implanted electrodes can be used for direct modulation of neural activity through electrical stimulation. In order to better understand the underlying mechanisms of stimulation, we first consider the effects of stimulation on a single neuron. At the single neuron level, the redistribution of charge, and subsequent depolarization, where the inside of the cell becomes more positive relative to the extracellular fluid, causes an action potential to propagate down an axon. Hyperpolarization, which occurs when the inside of the cell becomes more negative relative to the outside of the cell, can inhibit action potentials. Electrical stimulation causes a redistribution of charge within the axon, generating an action potential due to the diffusion of ions through sodium, potassium, and calcium channels  \cite{Bean2007}. 

In solutions, electrical stimulation results in the redistribution of ions through non-Faradaic reactions, and the transfer of electrons to electrolytes in the solution through Faradaic reactions \cite{Merrill2005}. There exist both reversible and irreversible Faradaic reactions, and which one occurs depends on the rates of the electron transfers relative to the mass transport of the reactant. Through these mechanisms, charge is redistributed. When this redistribution of charge causes depolarization directly beneath the electrode, the stimulation is referred to as cathodal stimulation, while electrical stimulation which causes hyperpolarization directly beneath the electrode is referred to as anodal stimulation. 

Stimulation on a local scale can be achieved through intracortical microstimulation (ICMS), where electrical stimulation activates neurons primarily through their axons passing through the region of cortex stimulated \cite{Tehovnik2006a,Nowak1998}. However, other regions of the cell such as the cell body and dendrites may also be activated depending on stimulus polarity and orientation. Anodal pulses best activate cell bodies and terminals, compared to cathodal pulses which best activate axons. In both cases, it is the outward flowing current at the axon initial segment or nodes of Ranvier along the axon that result in neuronal excitation \cite{Tehovnik2006a,McIntyre2000}. ICMS is thought to sparsely activate a population of cortical neurons, rather than just ones proximal to the stimulation electrode tip \cite{Histed2009}. 

Physiologically, electrical stimulation is thought to activate both inhibitory and excitatory populations of cells \cite{Butovas2003a}, and is not thought to evoke natural patterns of cortical activity \cite{Millard2015a}. Functional magnetic resonance imaging (fMRI) work along with microstimulation has demonstrated that microstimulation, at least in the visual cortical pathway, suppresses the output activity of neurons which have their afferents stimulated \cite{Logothetis2010}. 

DES of human cortex using larger electrodes, such as ECoG or DBS electrodes, produce much larger volumes of tissue activated through the application of a larger amount of current injected  \cite{Vincent2016} when compared to ICMS. Additionally, depending on the parameters of DES, stimulation can either evoke or inhibit activity \cite{Borchers2012}. For example, DES of language areas during a language task can disrupt speech production while DES of somatosensory cortex can evoke sensations and DES of motor cortex can evoke movements. In terms of subdural ECoG stimulation in humans, the patterns and types of cells activated are thought to depend on the intricate details of cortical geometry, cell fiber orientation \cite{Kudela2015}, and whether the pulses are anodal or cathodal \cite{Seo2015}. The mechanisms of DBS stimulation are not yet currently understood, and are thought to involve the modulation of the networks targeted by the stimulation, rather than solely immediate inhibitory effects on the targeted anatomic region \cite{Ashkan2017,Montgomery2008}.

In terms of subdural macroscale stimulation in our current stimulation protocols in human, the patterns and types of cells are thought to depend on the intricate details of cortical geometry, cell fiber orientation \cite{Kudela2015}, and whether the phases are anodal or cathodal. A model of subdural cortical stimulation \cite{Seo2015}, reported that neurons deeper in the bank (buried in cortex) are more activated during cathodal subdural stimulation, while those in the wider crown were activated during anodal stimulation. 

In total, this speaks to the immense complexities of engineering stimulation in humans, and the work that remains to be done in understanding both its physical effects and resulting neural ones. 

\subsection{Sensing}

A key part of a BCI is the recording of neural activity to use as a control signal in order to successfully modulate the system using stimulation. The summed activity of many hundreds of thousands of neurons in the cortical tissue under an ECoG electrode contributes to the electric voltage recorded from the electrode. The increased firing rate of populations of neurons results in a broad increase in power across all frequencies, which is more easily separable in the broadband gamma band (above 50 Hz), rather than the lower frequency bands \cite{Miller2008}. This is because other frequency bands modulate up and down independently during different tasks and brain states, masking the broadband increase in power. The higher frequency components are more asynchronous, and therefore are not as subject to this masking effect \cite{Hermes2017}. Lower frequency bands, such as the alpha and beta band, are thought to represent pulsed inhibition that serves to gate and coordinate neuronal firing \cite{Schalk2015}. Therefore, analysis of broadband gamma activity reveals the local neuronal firing dynamics, while analysis of theta, alpha, and beta frequency regimes yields insight into the coordinating mechanisms across the brain.

These different oscillatory features have been explored for advancing our understanding of how different cortical regions function during motor movement and language function \cite{Flint2017,Bouchard2013}. Measurements of these signals during motor and speech imagery have been employed in brain-computer interfaces to drive end effectors such as computer cursors \cite{Leuthardt2004,Leuthardt2006,Leuthardt2006a} . Physical interfaces with the world have been realized with ECoG-BCIs driving robotic arms \cite{Hotson2016}. Furthermore, non-motor regions can be used to drive ECoG-BCIs, illustrating the general utility of oscillatory band driven BCIs \cite{Ramsey2006,Wilson2006}.

\section{Current Clinical Uses of Direct Electrical Stimulation}

\subsection{Functional Mapping for Epilepsy and Tumors}

As detailed previously, DES is frequently used both intraoperatively and during a patient’s stay at the hospital for functional mapping and identifying areas of cortex associated with important cognitive functions \cite{Ojemann1989,Berger1989,Berger1992}. These mapping procedures are done both for epilepsy surgeries and tumor resections \cite{Ojemann1989,Berger1989,Berger1992}. Clinicians, using implanted ECoG electrodes or stimulators in the operating room, apply DES to various cortical and subcortical structures and pathways, and observe location dependent effects, including speech arrest in language regions, motor movements in motor cortex, and sensory percepts in somatosensory cortex. The results of these stimulation studies inform where the surgeons will plan to resect; for example, if the seizure focus is close to a language region, the surgeon and patient may decide the surgery is not worth the risk of a permanent language deficit. 

\subsection{Deep Brain Stimulation (DBS)}
	
Deep brain stimulation (DBS) is a prominent example of electrical stimulation of the brain. It is currently being used for therapeutic treatment of movement disorders (Parkinson’s disease \cite{Bronstein2011} and Essential Tremor \cite{DellaFlora2010}), and is also being explored for treating psychiatric illnesses (post-traumatic stress disorder, depression, obsessive compulsive disorder, Tourette syndrome \cite{Schrock2015}) and epilepsy treatment. Traditionally, linear probes of cylindrical contacts are inserted into deep brain structures such as the globus pallidus internus (GPi), subthalamic nucleus (STN), or ventral intermediate nucleus of the thalamus (VIm). Following implantation, clinicians may either be guided by intraoperative CT imaging, or wake the patient up intraoperatively to test for adverse effects of stimulation on different contacts, using a monopolar (one stimulating electrode and a distant return electrode), bipolar (two similarly sized electrodes), or multipolar arrangement of electrodes for the steering of current. 

Advances in BCI related to deep brain stimulation (DBS) have explored the use of closed loop DBS to trigger stimulation of deep brain structures in response to signals recorded from the surface of the cortex \cite{Herron2017}. Herron et al. used threshold crossing in the beta-band regime of recorded ECoG signals as a control decision to trigger DBS stimulation. This enables volitional control of DBS stimulation solely through recorded neural signals. Such closed-loop control of stimulation conserves power and helps extend the life of the DBS device, reducing the number of replacement surgeries needed over the life of a user. 

DBS is also being explored for the treatment of particular types of epilepsy. Partial onset seizures often spread through the circuitry of the basal ganglia, and therefore could be controlled using DBS strategies similar to those used for movement disorders \cite{Lega2010,Halpern2008}. 

\subsection{Closed Loop Stimulation for Epilepsy}

Closed loop stimulation of seizure foci is currently clinically available through the Neuropace RNS system \cite{Morrell2011,Lee2015}. A neurosurgeon implants ECoG electrodes either on the cortical surface or in deeper structures near the putative seizure focus. If an impending seizure is detected, high frequency stimulation is triggered near the seizure focus to control the seizure. This is a demonstration of clinically effective and already implemented DES in an ECoG BCI, where neural control signals are acquired in real time from the brain and used to trigger stimulation. 

\section{Advantages of DES Relative to Other Stimulation Techniques}

An advantage of DES relative to non-invasive electrical stimulation modalities is the much greater delivery to neurons of the applied current. During transcranial electrical stimulation (TES), as much as 75\% of the current is shunted through the scalp and the skull \cite{Widge2018,Voroslakos2018}. This greatly blunts the efficacy of cortical stimulation, and suggests that some of the published results using TES are due to mechanisms other than direct neuronal excitation. In contrast, by directly stimulating the brain and bypassing the skull and scalp, DES delivers current to cortical structures more effectively.

Even with epidural and subdural stimulation, not all current reaches neurons in the cortex. Epidural stimulation results in current shunting by the dura \cite{Wongsarnpigoon2008}, while both epidural and subdural stimulation have some degree of cerebrospinal fluid (CSF) shunting depending on the characteristics of the CSF beneath or surrounding the electrodes \cite{Wongsarnpigoon2008,Guler2018}. 

TMS has primarily been used to induce motor movements, rather than isolated sensory percepts (although tapping sensations and auditory clicks can accompany TMS) \cite{Sliwinska2014}. A method such as DES affords the ability to focally and specifically produce sensations that would not be achievable through TMS. 

Additionally, traditional figure-8 TMS coils are currently unable to target cortical structures beyond 2-3 cm deep \cite{Wagner2009,Roth1991}. DES electrodes, on the other hand, can be physically placed near deeper cortical regions of interest in order to elicit the desired stimulation effects. Another advantage of DES over TMS is the fact that the maximum of the electric field strength induced by TMS has to occur at the cortical surface, and not in deeper structures \cite{Heller1992}. This means that off-target effects in cortical layers near the surface are possible when targeting deeper structures. Even with more sophisticated coils, such as the H-coil, the maximum stimulation strength still occurs at the surface and greater depth of stimulation (4-6 cm) is achieved with a loss of focality \cite{Wagner2009,Zangen2005}. 

The fact that electrodes can be placed near the deeper structures of interest is vital for the treatment of Parkinson’s and Essential Tremor through DBS. As these structures cannot currently be effectively stimulated through alternative methods, effective clinical treatment relies upon the ability to place electrodes near the desired brain regions.  

\section{Research Directions for DES}

\subsection{Sensory Feedback through DES}

% this is from the intro to the RT paper
Integration of somatosensory feedback into brain-computer interfaces (BCIs) has been shown to improve BCI task performance \cite{Pistohl2015,Dadarlat2015,Klaes2014,Suminski2010,Schiefer2016a} and is also a consumer design priority for prosthetics users\cite{Biddiss2007a} and potential BCI end users such as individuals with paralysis\cite{Collinger2013,Anderson2004}. The study of cortical stimulation for providing somatosensory task feedback has garnered increasing attention because of the realization that the absence of sensory feedback in many current BCIs may limit performance and extensibility\cite{Bensmaia2014a}. Prior work has shown that humans can respond to direct electrical stimulation (DES) of the surface of the primary somatosensory (S1) cortex\cite{Johnson2013a,Hiremath2017,LIBET1964,Ray1999a}, which engenders an artificial sensory percept organized according to standard somatotopy. Recent work has revealed that S1 DES can be used for somatosensory feedback for closed-loop control in a motor task\cite{Cronin2016a}. Furthermore, DES has also been shown to induce prosthetic hand ownership\cite{Collins2016}. Thus, DES offers the potential to close the loop in human BCIs by providing a mechanism to encode sensory feedback from an end effector to a user.

With recent advances in materials and manufacturing, spatially smaller microECoG arrays are able to target smaller volumes of cortex. More targeted DES through microECoG grids demonstrates higher spatial selectivity relative to larger clinical electrode grids (Hiremath et al. 2017; Lee et al. 2018), opening up the possibility of encoding more complex percepts compared to larger electrodes.

A BCI application with DES (Figure \ref{fig:introClosedLoopS1})could integrate both behavioral output as well as cortical signatures of salient activity, such as reaching or grasping, to trigger stimulation of sensory cortex, resulting in a closed-loop sensorimotor BCI in humans. The recent demonstrations of usable sensory signals in humans via DES brings us a step closer to such closed-loop human BCIs.

\begin{figure}[ht]
	\centering
	\includegraphics[width=0.8\textwidth]{figures/introduction/closed_loop_ecog}
	\caption[Experimental protocol]{Neural signals are recorded from cortical regions, along with behavioral input, and are used to trigger stimulation of other cortical regions. This general procedure could be used for neural rehabilitation following injury such as stroke, as well as restoration of sensation for neuroprosthetic limbs.}
	\label{fig:introClosedLoopS1}
\end{figure}

\subsection{Quantification of Cortical Connectivity}

An additional application of DES is in quantifying cortical connectivity. DES of a cortical site can produce a cortico-cortical evoked potential (CCEP) at local and remote sites depending on the cortical area stimulated and the intensity of stimulation \cite{Keller2014e}. Studies have explored CCEPs in the context of different cortical networks, including language \cite{Matsumoto2004b} and motor regions \cite{Matsumoto2006a}. The connections probed with CCEPs correspond well with known functional networks observed through fMRI as well as white matter pathways confirmed by diffusion tensor imaging (DTI) \cite{Keller2014d}. Such evoked potentials could have utility in BCI applications where depending on the presence or modulation of these CCEPs, an algorithmic decision could be made. 

\subsection{Modification of Cortical Excitability and Plasticity}

Another use for DES currently being explored is the induction of cortical plasticity. This refers to enhancement or other modification of connectivity between different cortical regions, which could aid in the recovery of individuals suffering from disrupted neuronal communication due to injuries such as stroke. 

A persistent theme in cortical connectivity is the idea of Hebbian plasticity, a type of synaptic plasticity first proposed by Donald Hebb in 1949 \cite{Hebb1949}: presynaptic firing of one neuron (site A) can strengthen the connection between it and a postsynaptic neuron (site B) that fires soon after A. Bi and Poo demonstrated a version of this plasticity rule, known as spike timing dependent plasticity (STDP), in rat hippocampal slice cells: consistent firing of a presynaptic cell (site A) within a time window of 20 ms before another postsynaptic cell (site B) led to a strengthened connection (LTP) from A to B, while B firing in a time window of 20 ms before A led to a weakened connection (LTD) \cite{Bi1998}. Both of these mechanisms were determined to be dependent on NMDA receptors. 

These principles have been applied in animal rehabilitation experiments, where triggering stimulation in somatosensory cortex several milliseconds after premotor cortex firing in animals that suffered from damage to motor cortex resulted in increased functional performance \cite{Guggenmos2013a}. In a non-human primate (NHP) model, DES delivered during beta oscillations during the depolarizing potential (negative peak as recorded through LFPs) caused potentiation of cortical connectivity, while stimuli delivered during the hyperpolarizing potential caused depression of cortical connectivity as assessed through cortically evoked potentials \cite{Zanos2018}. Other work has explored the use of paired-pulse paradigms in NHPs, where concurrent surface to depth stimulation at one site consistently followed stimulation at another site with a fixed time lag to modulated evoked potentials \cite{Seeman2017}. The optimal time lag for potentiation was found to be between 10-30 ms, with longer delays not resulting in potentiation. Only a fraction of the sites in this study were potentiated, and effects were often seen globally, illustrating the complex factors influencing cortical plasticity. 

Beyond work in animals, and importantly, for applications such as stroke rehabilitation, recent work has reported improvements in physiological measures of motor function with non-invasive stimulation such as movement triggered TMS compared to random TMS stimulation \cite{Buetefisch2011}. Adding further support to the importance of brain state dependent stimulation for rehabilitation is a recent study that demonstrated TMS delivery during movement-related beta-band (16-22 Hz) desynchronization caused a significant increase in corticospinal excitability, as evaluated through motor evoked potentials, lasting beyond the period of stimulation and depotentiation \cite{Kraus2016}. 

Keller et al. demonstrated that repetitive 10 Hz DES using subdural electrodes induced both potentiation and suppression in different cortical sites, depending on the baseline network characteristics \cite{Keller2018}. This suggests that plasticity can indeed be modulated through DES in humans, and that individual patient models of connectivity may inform the optimal sites to target to either enhance or decrease connection strength. 

An example BCI application (Figure \ref{fig:introClosedLoopDepth} could include an oscillatory feature at a surface electrode, such as activity in the beta band, driving stimulation at a deeper cortical structure to modify neural activity and connectivity. The structures targeted could include a region such as the hippocampus, which is important for memory. This activity dependent stimulation is similar to many of the activity-dependent DBS paradigms being explored for restoring motor function (Herron et al. 2017), and highlights the general applicability of neural activity dependent stimulation. 

\begin{figure}[ht]
	\centering
	\includegraphics[width=0.8\textwidth]{figures/introduction/closed_loop_DBS}
	\caption[BCI cortically driven closed loop stimulation.]{Current exploratory clinical uses for stimulation and closed loop BCIs include the measurement of cortical signals (such as power in beta oscillatory bands), and depending on a threshold, the subsequent decision of when to stimulate another cortical or subcortical target. We highlight here as a potential example, a decision to target deep cortical structures based off an increase in the beta band frequency on a surface ECoG electrode. }
	\label{fig:introClosedLoopDepth}
\end{figure}

The combination of theoretical, animal, and human data discussed above suggests that activity-dependent DES is a promising approach to enhance and modify connectivity in humans, offering a new type of therapy for targeted restoration of function after neural injury. ECoG BCIs are well-suited to acquiring and decoding appropriate control signals and when coupled with DES, can be used to influence cortical activity and induce activity-dependent plasticity. 

\section{Limitations and Considerations}

While ECoG based bi-directional BCIs offer several advantages over other types of BCIs, there are limitations and considerations that must be taken into account before any application. For either subdural or epidural electrodes, neurosurgery is required. The size of the electrodes, relative to other invasive methods such as ICMS, targets a larger population of neurons. Furthermore, there is no ability to target specific types of cells. The developing field of optogenetics \cite{Yizhar2011,Deisseroth2011} describes the use of genetic modification and optical methods to either activate or inactive specific neurons in-vivo. Although optogenetics may offer a more targeted approach to activating neurons, progress to humans may be slow due to the technique’s reliance on genetic modification of neurons. 

Another current consideration when developing technologies and protocols to induce plasticity is our current lack of understanding of the mechanisms of plasticity induction \cite{Feldman2012}. Beyond the single neuron spiking level, plasticity is a complex phenomenon as discussed above, and in a human brain, the potential factors influencing plasticity can be complex and numerous. 

Although DES may offer a promising approach to inducing plasticity, it has yet to be demonstrated to be unequivocally effective in a stroke model. Limited subgroups of stroke patients with residual motor function were shown to benefit from open-loop DES in the EVEREST trial, but other groups showed no benefit \cite{Levy2016a}. As better animal models of stroke are developed \cite{Sommer2017}, one can hope to gain a better mechanistic understanding of how DES can be used for stroke rehabilitation, leading to optimized therapies for maximizing functional recovery following cortical injury. 

The issue of particular patient subgroup benefit as discussed above speaks to the broader issue of patient variability. Due to anatomic or surgical variations, results from one group of subjects may not necessarily apply to another. Careful consideration of these individual factors will be important for future bidirectional BCIs. 

An additional consideration is the durability of electrodes with repeated stimulation. Charge transfer can occur through irreversible Faradaic reactions, where electrolysis occurs, and depending on the polarity of stimulation, either hydrogen gas or oxygen gas are the by-products \cite{Merrill2005}. In this electrolytic window, accelerated corrosion and electrode damage can occur. Even below the voltage required for the electrolysis of water, detrimental byproducts such as the formation of metal chloride and hydrogen peroxide can occur, leading to electrode corrosion. Therefore, long term use of stimulating ECoG electrodes will require careful selection of stimulation parameters and materials to minimize adverse effects. 

Beyond electrode damage, tissue damage induced by stimulation is a key consideration for long term use of DES. The study of electrical stimulation through platinum electrodes in cats \cite{McCreery1990} was used to define the Shannon equation \cite{Shannon1992}, which has been used frequently for assessing safe stimulation levels. Earlier research established a 30 $ \frac{\mu C}{cm2} $ limit on the charge per phase of stimulation for macro-scale electrodes (in particular, DBS electrodes) \cite{Kuncel2004}, but tissue damage can occur above and below this threshold \cite{Cogan2016b}. There are factors influencing whether or not tissue damage occurs that are not included in the Shannon equation, for example, the scale of the electrode (macro vs. micro), the current density, duty cycle, and pulse frequency, and the uniformity of current distribution \cite{Cogan2016b}. These complex factors will require further modeling and laboratory testing to establish what the appropriate stimulation parameters are to minimize tissue damage, particularly with the use of novel materials and stimulation patterns. 

With penetrating microelectrodes (such as with the Utah array), there is a significant change in the electrode-tissue interface over time \cite{Williams2007}. In addition, stimulation can change the characteristics of the electrode-tissue interface. A recent study analyzing the impedance characteristics of DBS electrodes following implantation and stimulation has shown that DBS electrode impedance increases after implantation and decreases with clinically relevant stimulation \cite{Lempka2009}. Other work has shown that the stimulation parameters used affect the impedance measured for DBS electrodes \cite{Wei2009}. ECoG electrode impedance measurements from 191 persons implanted with the Neuropace RNS system, over a median time of 802 days, did not reveal significant differences between stimulating and non-stimulating electrodes in peri-implant changes in impedance or impedance stability \cite{Ryapolova-Webb2014}. In this study, while there were statistically significant short-term changes in impedance following implantation, long-term impedances were stable. These results suggest that ECoG BCIs with concurrent DES may prove viable as chronic implants. 

In any closed-loop application involving concurrent stimulation and recording, the electrical artifact due to stimulation is many orders of magnitude greater than the neural signals being recorded. Disentangling the volume conduction of the stimulation pulse from the neural responses is a topic of active research \cite{Zhou2018}. Different approaches have been used for handling artifacts, ranging from hardware approaches to mitigate artifacts before signal acquisition to post-processing techniques to minimize artifacts after the signals have been acquired. 

\section{Relevant Neural System Physiology}

As this thesis is considered primarily with stimulation of motor and sensory systems, we will review the physiology of these two systems. These two systems are integrally related through sensorimotor integration, where motor output is modified sensory input \cite{Ackerley2015}.

\subsection{Motor System}
How do we volitionally reach out and interact with the world around us?

\subsection{Somatosensory System}
How do we acquire information about world around us? The human somatosensory system refers to 

\subsubsection{Peripheral Nervous System}

\subsubsection{Central Nervous System}


